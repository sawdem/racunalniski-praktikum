\documentclass[11pt]{article}
\usepackage[a4paper, margin=2.5cm]{geometry}
\usepackage{biblatex}
\usepackage{graphicx}
\usepackage{amsmath}

\addbibresource{knjiga.bib}

\title{Brownovo gibanje}
\author{Matej Rojec}
\date{}

\newtheorem{definition}{Definicija}
\newtheorem{theorem}{Izrek}

\begin{document}
\maketitle
\paragraph{}
Brownovo gibanje (več v \cite{karatzas1991brownian}) je intuitivno slučajen proces, % Sklic na knjigo
ki predstavlja naključno gibanje delcev v mediju.

\renewcommand{\figurename}{Slika}
% Slika: PerrinPlot2.pdf
\begin{figure}[h!]
  \centering
  \includegraphics[width=60mm]{PerrinPlot2.pdf}
         \caption{Reprodukcija slike iz Jean Baptiste Perrin, \emph{Mouvement brownien et réalité moléculaire}, Ann. de Chimie et de Physique (VIII) 18, 5-114, 1909}
       \end{figure}

    % Začetek definicije
   \begin{definition}
     \normalfont Standardno Brownovo gibanje $\{B_t\}_{t \geq 0}$ je slučajen proces z naslednjimi lastnostmi: 
         \begin{enumerate}
            \item $B_0 = 0$.
            \item Prirastki $B_{t_n} - B_{t_{n-1}}, B_{t_{n-1}} - B_{t_{n-2}}, \ldots, B_2 - B_1, B_1 - B_0$ so neodvisne slučajne spremenljivke, za vsak $t_0 \leq t_1 \leq \cdots \leq t_{n-1} \leq t_n$.
            \item Za vsak $t \geq 0$ in $h > 0$ velja $B_{t+h} - B_t \sim \mathcal{N}(0, h)$.
            \item Funkcija $t \mapsto B_t$ je zvezna skoraj gotovo.
         \end{enumerate}
   \end{definition}
    % Konec definicije
    
    Preden zapišemo izrek, definirajmo še pojem časa ustavljanja.
    
    % Začetek definicije
   \begin{definition}
     \normalfont Slučajna spremenljivka $\tau$ na verjetnostnem prostoru ($\Omega, \mathcal{F}, P$) z vrednostmi v %$\mathbb{"R"}$
     je \emph{čas ustavljanja} glede na filtracijo ??, če velja ??.
   \end{definition}
    % Konec definicije
    
    Zdaj lahko zapišemo izrek. % Sklic na izrek z oznako thm:stopped_brownian
    
    % Začetek izreka
    \begin{theorem}
      \normalfont Naj bo $\{B_t\}_{t \geq 0}$ (standardno) Brownovo gibanje, $\tau$ čas ustavljanja glede na 
      $\{\mathcal{F}_t\}_{t \geq 0}$ in naj velja \boldmath{P}$[\tau<\infty]$.
      Potem je tudi proces:
      \[
      \hat{B} := \{B_{T+t} - B_T \mid t \geq 0\}
      \]
      (standardno) Brownovo gibanje in neodvisen od $\mathcal{F}_T$.
    \end{theorem}
    % Konec izreka
    \printbibliography[title={Literatura}]
\end{document}